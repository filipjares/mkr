\RequirePackage{ifpdf}
%\documentclass[notes,handout]{beamer}
\documentclass[notes=false,pdftex]{beamer}
\usepackage[latin1,utf8]{inputenc}
\usepackage[czech]{babel}
\usepackage{multimedia}
\usepackage{listings}
\lstset{
	language=C,
	basicstyle=\small,
	tabsize=2,
	numbers=left,             
	numberstyle=\scriptsize,  
	stepnumber=1,             
	numbersep=10pt            
}

\usetheme{umbc2}
%\useoutertheme{umbcfootline} 

\author{Filip Jareš, Tomáš Juchelka} 
\title{Segmentace polygonální mapy pro multirobotickou exploraci} 
\institute[FEL ČVUT] {
  Katedra kybernetiky\\[2ex]
  Elektrotechnická fakulta\\[2ex]
  České vysoké učení technické\\[2ex]
  \vspace{1ex}
  \texttt{jaresfil@fel.cvut.cz, juchetom@fel.cvut.cz}
  \vspace{1ex}
}
\date{Mobilní a kolektivní robotika (A3M33MKR)\hspace{7ex}18. 1. 2012}

%\setbeamertemplate{items}[ball] 
%\setbeamertemplate{navigation symbols}{} 
%\setbeamertemplate{blocks}[rounded][shadow=true] 
\setbeamercovered{transparent}

\begin{document}

%%%%%%%%%%%%%%%%%%%%%%%%%%%%%%%% Titulní slide %%%%%%%%%%%%%%%%%%%%%%%%%%%%%%%%%%%%%%%%%%%%%%%%%%
\begin{frame}[plain]
  \titlepage
\end{frame}

% \section*{Obsah}
% \begin{frame}
% 	\frametitle{Obsah prezentace}
% 	\tableofcontents
% \end{frame}

%%%%%%%%%%%%%%%%%%%%%%%%%%%%%%%% Přehled %%%%%%%%%%%%%%%%%%%%%%%%%%%%%%%%%%%%%%%%%%%%%%%%%%%%%%%%

\section*{Úvod}
\begin{frame}
	\frametitle{Úvod}

	Využití segmentace

	\begin{itemize}
		\item multirobotická explorace prostředí
		\item efektivní přidělování cílů jednotlivým robotům
		\item \uv{rovnoměrné} rozptýlení robotů v prostředí
	\end{itemize}

	\includegraphics[width=1\textwidth]{images/TRACLabs_scan-clipped.jpg} 

\end{frame}

%%%%%%%%%%%%%%%%%%%%%%%%%%%%%%%% Název sekce %%%%%%%%%%%%%%%%%%%%%%%%%%%%%%%%%%%%%%%%%%%%%%%%%%%%

\section{Tvorba polygonu}
{
\begin{frame}% [plain] % plain znamena, bez standardniho pozadi, ktere maji ostanti slidy
	\frametitle{Tvorba polygonu}

	\begin{itemize}
		\item data ze senzoru ve formě vektoru vzdáleností
		\item výsledný scan $\longrightarrow$ množina bodů vs. polygon
	\end{itemize}

\end{frame}
}

%%%%%%%%%%%%%%%%%%%%%%%%%%%%%%%% Název sekce %%%%%%%%%%%%%%%%%%%%%%%%%%%%%%%%%%%%%%%%%%%%%%%%%%%%

\section{Tvorba mapy}
\begin{frame}
	\frametitle{Tvorba polygonu}

	\begin{columns}[T]
		\begin{column}{0.6\textwidth}
			\begin{itemize}
				\item k dispozici polygon ze senzorického měření \\
				\item booleovské operace nad polygony \\
				\item využití knihovny pro práci s~polygony:
					Clipper 4.6.3 -- an~open source polygon clipping library
					\cite{ClipperLib}
			\end{itemize}
		\end{column}
		\begin{column}{0.4\textwidth}
			\includegraphics[width=1\textwidth]{images/clipper.png} 
		\end{column}
	\end{columns}

\end{frame}
%%%%%%%%%%%%%%%%%%%%%%%%%%%%%%%% Název sekce %%%%%%%%%%%%%%%%%%%%%%%%%%%%%%%%%%%%%%%%%%%%%%%%%%%%

\section{Voroného diagram}
\begin{frame}
	\frametitle{Voroného diagram, topologická kostra polygonální mapy}

	\begin{columns}[T]
		\begin{column}{0.5\textwidth}
			\begin{itemize}
				\item VD(S)
			\end{itemize}
		\end{column}
		\begin{column}{0.5\textwidth}
			\includegraphics[width=1\textwidth,clip=true,trim=147pt 165pt 289pt 502pt]{images/vd_01.pdf} 
		\end{column}
	\end{columns}
\end{frame}
\begin{frame}
	\frametitle{Voroného diagram, topologická kostra polygonální mapy}

	\begin{columns}[T]
		\begin{column}{0.5\textwidth}
			\begin{itemize}
				\item VD(S)
			\end{itemize}
		\end{column}
		\begin{column}{0.5\textwidth}
			\includegraphics[width=1\textwidth,clip=true,trim=147pt 165pt 289pt 502pt]{images/vd_02.pdf} 
		\end{column}
	\end{columns}
\end{frame}
\begin{frame}
	\frametitle{Voroného diagram, topologická kostra polygonální mapy}

	\begin{columns}[T]
		\begin{column}{0.5\textwidth}
			\begin{itemize}
				\item VD(S)
			\end{itemize}
		\end{column}
		\begin{column}{0.5\textwidth}
			\includegraphics[width=1\textwidth,clip=true,trim=147pt 165pt 289pt 502pt]{images/vd_03.pdf} 
		\end{column}
	\end{columns}
\end{frame}
\begin{frame}
	\frametitle{Voroného diagram, topologická kostra polygonální mapy}

	\begin{columns}[T]
		\begin{column}{0.5\textwidth}
			\begin{itemize}
				\item VD(S)
			\end{itemize}
		\end{column}
		\begin{column}{0.5\textwidth}
			\includegraphics[width=1\textwidth,clip=true,trim=147pt 165pt 289pt 502pt]{images/vd_04.pdf} 
		\end{column}
	\end{columns}
\end{frame}

%%%%%%%%%%%%%%%%%%%%%%%%%%%%%%%% Název sekce %%%%%%%%%%%%%%%%%%%%%%%%%%%%%%%%%%%%%%%%%%%%%%%%%%%%

% Tahle stránka obsahuje odkaz (\hyperlink{diagram}) na jinou stránku, která má
% přiřazený label "diagram"
\begin{frame}[fragile,label=code]
	\frametitle{Příklad s kódem, který se postupně odkrývá}
\note{
	\begin{itemize}
		\item definice stavové funkce stavu \texttt{wait\_for\_command}
		\item komentář ...
	\end{itemize}
}

	\small{
	\begin{semiverbatim}
	\uncover<1->{FSM_STATE(wait_for_command) \{ }
	\uncover<1->{  switch (FSM_EVENT) \{ }
	\uncover<2->{  case EV_ENTRY: // zde zajistime uvedeni mechanismu }
	\uncover<2->{  	  break;      // do transportni polohy }
	\uncover<3->{  case EV_LOAD_THE_PUCK: FSM_TRANSITION(load_the_puck); }
	\uncover<3->{  	  break; }
	\uncover<4->{  case EV_PREPARE_THE_UNLOAD: }
	\uncover<4->{  	  if (robot.pucks_inside == 0) \{ }
	\uncover<4->{  	  	  FSM_SIGNAL(MAIN, EV_ACTION_ERROR, NULL); }
	\uncover<4->{  	  \} }\uncover<5->{ else \{ }
	\uncover<5->{  	  	  floor_to_unload = FSM_EVENT_INT; }
	\uncover<5->{  	  	  FSM_TRANSITION(unload_pucks); }
	\uncover<5->{  	  \} }
	\uncover<5->{  	  break; }
	\uncover<6->{  case EV_EXIT: break; }
	\uncover<6->{  \} }
	\uncover<6->{\} } \hspace{\fill} \hyperlink{diagram}{\beamerskipbutton{odkaz na jinou stránku}}
	\end{semiverbatim}
	}

\end{frame}

%%%%%%%%%%%%%%%%%%%%%%%%%%%%%%%% Hlavni FSM %%%%%%%%%%%%%%%%%%%%%%%%%%%%%%%%%%%%%%%%%%%%%%%%%%%%%

\section{Stavový automat pro řízení herní strategie}
\begin{frame}
	\frametitle{Stavový automat pro řízení herní strategie}

\note{
	\begin{itemize}
		\item kombinaci puků získává od programu \emph{rozpuk}
	\end{itemize}
}
	%\includegraphics[width=1\textwidth]{pics/competition.pdf} 

	
\end{frame}

%%%%%%%%%%%%%%%%%%%%%%%%%%%%%%%% Mechanismy robotu %%%%%%%%%%%%%%%%%%%%%%%%%%%%%%%%%%%%%%%%%%%%%%

\section{Stavový automat mechanismů}
\begin{frame}
	\frametitle{Mechanismy robotu pro manipulaci s puky}
\note{
	\begin{itemize}
		\item zmínit zde senzor -- koncový spínač
	\end{itemize}

}

	\begin{columns}[T]
		\begin{column}{0.75\textwidth}
			%\includegraphics[width=1\textwidth]{images/ilustrace/mechanismy.pdf} 
		\end{column}
		\begin{column}{0.25\textwidth}
			\begin{itemize}
				\item[1.] čelisti
				\item[2.] pásy
				\item[3.] výtah
				\item[4.] dvířka
				\item[5.] vytlačovač
			\end{itemize}
		\end{column}
	\end{columns}

\end{frame}

%%%%%%%%%%%%%%%%%%%%%%%%%%%%%%%% Automat mechanismu %%%%%%%%%%%%%%%%%%%%%%%%%%%%%%%%%%%%%%%%%%%%%

\begin{frame}[label=diagram]
	\frametitle{Stavový automat mechanismů}

	%\includegraphics[width=1\textwidth]{pics/fsmact.pdf} 
	\vspace{3ex}
	\centering
	\hyperlink{code}{\beamerreturnbutton{definice wait\_for\_command}}
	
\end{frame}

%%%%%%%%%%%%%%%%%%%%%%%%%%%%%%%% Závěr %%%%%%%%%%%%%%%%%%%%%%%%%%%%%%%%%%%%%%%%%%%%%%%%%%%%%%%%%%

\section{Závěr}
\begin{frame}
	% tohle byl pokus zahrnout odkaz na video, ale byly s tím spíš problémy
	% \frametitle{\movie[externalviewer]{Závěr}{robojizda.avi}}
	\frametitle{Závěr}

	\begin{columns}[T]
		\begin{column}{0.5\textwidth}
			%\includegraphics[width=1\textwidth]{pics/robot-vyrez.jpg} 
		\end{column}
		\begin{column}{0.5\textwidth}
			\begin{itemize}
				\item body k tomu, co by šlo udělat lépe\pause
				\item Citace, která se zobrazí v referencích: \cite{Wurm2008Coordinated}
				\item Clipper, The polygon clipping library: \cite{ClipperLib}
			\end{itemize}
			\pause
			\vspace{13ex}\hspace*{\fill}Děkujeme za pozornost
		\end{column}
	\end{columns}

\end{frame}

\section{Reference}
\begin{frame}
	% tohle byl pokus zahrnout odkaz na video, ale byly s tím spíš problémy
	% \frametitle{\movie[externalviewer]{Závěr}{robojizda.avi}}
	\frametitle{Závěr}

	% The Bibliography
	%\bibliographystyle{plain_cz}
	%\bibliographystyle{czechiso}
	\bibliographystyle{ieeetr}
	\bibliography{bibliography}
	
\end{frame}



\end{document}

